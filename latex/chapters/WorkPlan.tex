\chapter{Work Plan}

\section{Work Plan Timeline}

\newacronym{pdr}{PDR}{Preliminary Dissertation Report}
\newacronym{soa}{SOA}{State of the Art}


The work plan for this thesis is structured in three different phases. 

The first phase warranted a background study of the ZX-calculus and the \textit{PyZX} tool. This study encompassed the specification and simplification of quantum circuits. It concluded with the writing of the \acrshort{pdr}.

The second phase pertains to the identification of of case applications in the domain of quantum machine learning. That is, a brief study of circuit training in quantum machine learning. Along with an understanding of the barren plateau problem and other related phenomena.

The final phase includes a detailed analysis of the cases identified in the previous phase, as well as a documentation of these. The final months will be dedicated to the writing up of the dissertation.


\begin{table}[H]
\begin{center}
\begin{tabular}{| c | c | c | c | c | c | c | c | c | c | c |}
\hline
\textbf{Task} & \textbf{Oct} & \textbf{Nov} & \textbf{Dec} & \textbf{Jan} & \textbf{Feb} & \textbf{Mar} & \textbf{Apr} & \textbf{May} & \textbf{Jun} & \textbf{Jul}\\
\hline
Background and \acrshort{soa} & $\bullet$ & $\bullet$ & $\bullet$ &$\bullet$ &$\bullet$ & & & & & \\
\hline
\acrshort{pdr} preparation & &  & $\bullet$ & $\bullet$ & & & & & & \\
\hline
Modelling in ZX & & & & &$\bullet$ &$\bullet$ &$\bullet$ &$\bullet$ &$\bullet$ & \\
\hline
Establishing properties & & & & &$\bullet$ &$\bullet$ &$\bullet$ &$\bullet$ &$\bullet$ & \\
\hline
Writing Dissertation & & & & & & & $\bullet$ & $\bullet$ & $\bullet$ & $\bullet$ \\
\hline
\end{tabular}
\end{center}
\caption{Work Plan Timeline}
\end{table}


The table above establishes a time frame for each of the distinct tasks pertaining to this thesis. Although these tasks do not necessarily fit in just one phase, as many of them overlap different phases.

\section{Task Details}
\subsection{Background and SOA}
\label{back-soa}
This task encompasses all the background study that being the study ZX-calculus, quantum machine learning, specifying and simplifying quantum circuits.

Along with this it also concerns a study on the state of the art of the different relevant fields.

\subsection{PDR preparation}

This task involves the writing of the present document.

\subsection{Modelling in ZX}

This task along with \ref{Est. prop} is the core of this dissertation. In it quantum algorithms, identified in \ref{back-soa}, will be modelled as ZX-diagrams and reasoned with. Potentially leading to the optimization of the circuits and protocols used. 

\subsection{Establishing properties}
\label{Est. prop}

In tandem with the previous task relevant properties may arise. As such, this task aims to identify and establish these properties.


\subsection{Writing Dissertation}

This is the final task in the work plan. It simply relates to the writing of the dissertation.

